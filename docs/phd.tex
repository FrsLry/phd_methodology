%%%%%%%%%%%%%%%%%%%%%%%%%%%%%%%%%%%%%%%%%%%%%%%%

% Specify the command that you want into the header of the
% index.md file

%%%%%%%%%%%%%%%%%%%%%%%%%%%%%%%%%%%%%%%%%%%%%%%%

% Options for packages loaded elsewhere
\PassOptionsToPackage{unicode}{hyperref}
\PassOptionsToPackage{hyphens}{url}
\PassOptionsToPackage{dvipsnames,svgnames*,x11names*}{xcolor}
%
\documentclass[
  12pt,
  oneside]{report}
%%\usepackage{lmodern}
%
% Set line spacing
\usepackage{setspace}
\setstretch{1.5}

\usepackage{amssymb,amsmath}
\usepackage{ifxetex,ifluatex}
\ifnum 0\ifxetex 1\fi\ifluatex 1\fi=0 % if pdftex
  \usepackage[T1]{fontenc}
  \usepackage[utf8]{inputenc}
  \usepackage{textcomp} % provide euro and other symbols
\else % if luatex or xetex
  \usepackage{unicode-math}
  \defaultfontfeatures{Scale=MatchLowercase}
  \defaultfontfeatures[\rmfamily]{Ligatures=TeX,Scale=1}
\fi
% Use upquote if available, for straight quotes in verbatim environments
\IfFileExists{upquote.sty}{\usepackage{upquote}}{}
\IfFileExists{microtype.sty}{% use microtype if available
  \usepackage[]{microtype}
  \UseMicrotypeSet[protrusion]{basicmath} % disable protrusion for tt fonts
}{}
\makeatletter
\@ifundefined{KOMAClassName}{% if non-KOMA class
  \IfFileExists{parskip.sty}{%
    \usepackage{parskip}
  }{% else
    \setlength{\parindent}{0pt}
    \setlength{\parskip}{6pt plus 2pt minus 1pt}}
}{% if KOMA class
  \KOMAoptions{parskip=half}}
\makeatother
\usepackage{xcolor}
\IfFileExists{xurl.sty}{\usepackage{xurl}}{} % add URL line breaks if available
\IfFileExists{bookmark.sty}{\usepackage{bookmark}}{\usepackage{hyperref}}
\hypersetup{
  pdftitle={PhD Methodology},
  pdfauthor={François Leroy},
  colorlinks=true,
  linkcolor=Maroon,
  filecolor=Maroon,
  citecolor=Blue,
  urlcolor=Blue,
  pdfcreator={LaTeX via pandoc}}
\urlstyle{same} % disable monospaced font for URLs

%% Package geometry
\usepackage[left = 2cm,right = 2cm,top = 2cm,bottom = 2cm]{geometry}
\usepackage{pdflscape}


\usepackage{longtable,booktabs}
% Correct order of tables after \paragraph or \subparagraph
\usepackage{etoolbox}
\makeatletter
\patchcmd\longtable{\par}{\if@noskipsec\mbox{}\fi\par}{}{}
\makeatother
% Allow footnotes in longtable head/foot
\IfFileExists{footnotehyper.sty}{\usepackage{footnotehyper}}{\usepackage{footnote}}
\makesavenoteenv{longtable}
\setlength{\emergencystretch}{3em} % prevent overfull lines
\providecommand{\tightlist}{%
  \setlength{\itemsep}{0pt}\setlength{\parskip}{0pt}}
\setcounter{secnumdepth}{5}
%%% Complete the preamble of the LaTeX template
%%%------------------------------------------------------------------------------

%% Bug de bookdown: ne traite plus la déclaration "otherlangs" dans le préambule
% Pour charger les langues, écriture ici en dur du produit de bookdown
% Corrigé le 22/11/2019. A retester régulièrement: supprimer ces lignes si la compilation fonctionne sans elles.
\usepackage{polyglossia}
  \setmainlanguage[variant=american]{english}
  \setotherlanguage[]{french}
% Bug persistant le 28/02/2020

% Advised with polyglossia and babel
\usepackage{csquotes}

% Environnement "Essentiel" en début de chapitre
\usepackage[tikz]{bclogo}
\newenvironment{Essentiel}
  {\begin{bclogo}[logo=\bctrombone, noborder=true, couleur=lightgray!50]{L'essentiel}\parindent0pt}
  {\end{bclogo}}

%% Package fontspec
\usepackage{fontspec}
\setmainfont{calibri}[
  Path           = ./fonts/,
  Extension      = .ttf,
  BoldFont       = calibrib,
  ItalicFont     = calibrili,
  BoldItalicFont = calibriz]

% Rename chapters
% Below, scrpit to prevent the "chapter n" and the space use for it to
% be displayed
\usepackage{titlesec}
\titleformat{\chapter}   
{\Huge}{\thechapter{. }}{0pt}{\Huge}
%{\thechapter{. }}
\titlespacing*{\chapter}{0pt}{-50pt}{10pt}
% -50 is to up the title and 10 is the space with the text below



\ifluatex
  \usepackage{selnolig}  % disable illegal ligatures
\fi
\usepackage[style=apa,]{biblatex}
\addbibresource{references.bib}

\title{PhD Methodology}
\author{François Leroy}
\date{2021-04-06}

% to include pdf
\usepackage{pdfpages}



%%%%%%%%%%%%%%%%%%%%%%%%%%%%%%%%%%%%%%%%%%%%%%%%%%%%%%%%%%%%%
% Start of the documents
\begin{document}
\maketitle


% Roman numbering for content before toc and toc itself
\cleardoublepage 
\pagenumbering{roman}

{
\hypersetup{linkcolor=}
\setcounter{tocdepth}{1}
\tableofcontents
}
\listoffigures
\listoftables
\setstretch{1.5}


% Start the arabic numbering at the 1st chapter
\cleardoublepage 
\pagenumbering{arabic}


% The mind, the...
\hypertarget{annotation}{%
\chapter*{Annotation}\label{annotation}}
\addcontentsline{toc}{chapter}{Annotation}

Biodiversity, at the basis of many essentials ecosystem services, is in the process of facing its sixth mass extinction. Although global extinction is unprecedented, there is so far no reason to expect that biodiversity dynamic at lower spatial and temporal scales follow this trend. Thus, links between
spatio-temporal scales and facets of biodiversity (\emph{i.e.} species richness, species diversity, colonization, extinction,
species turnover, etc) need to be fully understood if we want to address this worldwide crisis. So far,
attempts to describe biodiversity changes have been limited mainly by heterogeneity in spatial and
temporal scales that was hardly taken into account by the statistical modelling frameworks.

My PhD project propose to address this flaws in order to understand in more details biodiversity
changes across spatial and temporal scales. Especially, we aim at developing and testing nonparametric
tree-based modelling methods allowing to study the non-linear and interacting effects of
space and time-span on different aspects of biodiversity.

\underline{The specific objectives of my PhD project are:}

\begin{enumerate}
\def\labelenumi{\arabic{enumi}.}
\tightlist
\item
  Modelling and mapping avian species richness changes over Czech Republic across space and time
  scales.
\item
  Decompose the modelled biodiversity to colonization, extinction, species turnover, across spatiotemporal
  scales.
\item
  Estimate the strength of the link between environmental drivers of biodiversity change across
  spatio-temporal scales.
\item
  Apply the previously developed method to other European regions (e.g.~UK, Switzerland, France)
\end{enumerate}

\hypertarget{intro}{%
\chapter{Introduction}\label{intro}}

Human life quality is intrinsically linked to ecosystems state that he is living in. Indeed, ecosystems services extend in a large spectrum of mechanisms including nutrient cycle, food production, or climate and water cycle regulation \autocite{pereira_global_2012}. Some of those ecosystem
functions are managed by bird populations such as seed dispersal, controls pests or pollinate plant.
Unfortunately, anthropogenic stressors like habitat loss, over exploitation, pollution or introduction of
invasive species could lead biodiversity to its sixth mass extinction \autocite{barnosky_has_2011}.

While the loss of global biodiversity is unprecedented, current scientific literature has also shown that
temporal trends in local changes of biodiversity can be opposite to trends at larger scales \autocite{chase_species_2019}. Thus, current changes in biodiversity is far more complex than a simple global decrease:
most of the ecosystems undergo alterations of their communities with changes in species composition \autocite{blowes_geography_2019,dornelas_assemblage_2014}.

Typically, biodiversity is considered for a particular taxon (\emph{e.g.} birds, amphibians, reptiles\ldots), but also
according to the spatial scale it is defined by. Here, the term scale refers to the area in which the
biodiversity in considered, also referred hereafter as grain size. So far, it has been assumed that
holding the spatial scale constant when studying biodiversity is mandatory \autocite{whittaker_scale_2001}. As
a matter of fact, it is known that species richness increases with the area considered \autocite{arrhenius_species_1921}.

and this relationship is approximately linear on a log-log scale (Species-Area Relationship,
SAR). However, this assumption restricts the data accessibility as sampling plans widely differ
according to the species studied, the resources available or, the field conditions. Thus, developing a
method capable of dealing with biodiversity across varying grain size could increase significantly the
data availability. Moreover, it would allow to model biodiversity at different spatial scales than the
ones used in the data. Modelling biodiversity indexes at finer spatial grain size that the data used to
learn the model is referred as \emph{downscaling} biodiversity whilst extrapolating at coarser grain-size is
called \emph{upscaling}.

So far, there are indications that such method can be used. For instance, \textcite{keil_downscaling_2014} and \textcite{keil_downscaling_2013} showed promising downscaling biodiversity models using biodiversity data with different
spatial scales, whilst \textcite{kunin_upscaling_2018} showed that upscaling biodiversity is also possible. Thus, all
the constituents of cross-scales models are known but still need to be gathered and tested. For
instance, \textcite{jarzyna_spatial_2015} used a Bayesian framework to study temporal changes of avian
biodiversity (colonization, extinction, temporal turnover) across scales. However, other approaches
such as parametric Generalized Linear Models (GLM), Generalized Additive Models (GAM) and
Generalized Linear Mixed Model (GLMM) or non-parametric tree based machine learning methods
need to be tested.


\singlespacing % reset the spacing of the bibliography style
%%%%%%%%%%%%%% Here is the part that I am using for the bibliography to be displayed in the toc
% First step: I define the name and label of the biblio part
\chapter{References}\label{references}
{
% I temporarily redefine the clearpage in order for the bib to not be printed on a new page
\renewcommand{\clearpage}{}
\printbibliography[heading=none] % I delete the default name of the bib
%\printbibliography 
}

\end{document}
